\documentclass[../main.tex]{subfiles}

\begin{document}

    \subsection{Experimento 1: Determinación del orden de la reacción.}
    \begin{enumerate}
    \item Se ubican los 6 tubos de ensayo vacíos que se pueden identificar con un numero y una letra (1A, 2A, 3A, 1B, 2B y 3B).
    \item A continuación se colocará cierta cantidad de militros de $K_2$$S_2$$O_8$ 1 M a cada tubo de ensayo con la letra A (2.5 mL al tubo con la etiqueta 1A, 2.5 mL al tubo de etiqueta 2A y 1.5 mL al tubo de etiqueta 3A).
    \item Después, se colocará 5 mL del indicador (almidón 2\%) más tiosulfato de sodio 0.0001 M  a cada tubo de ensayo con la letra B.
    \item  Luego, agregue cierta cantidad de mililitro de $KI$ 0,1 M y agua acada tubo de ensayo B (2.5 mL de $KI$ 0,1 M y 0.0 mL de $H_2O$  al tubo con la etiqueda 1B,  2.0 mL de $KI$ 0,1 M y 0.5 mL de $H_2O$  al tubo con la etiqueda 2B y 2.0 mL de $KI$ 0,1 M y 1.0 mL de $H_2O$  al tubo con la etiqueda 3B).
    \item Posteriormente, se juntará las soluciones de los tubos con el mismo número (1A con 1B, 2A con 2B y 3A con 3B) y se tomará el tiempo en el hasta que la mezcla se torne de color azul.
    \item Por consiguiente, se anota el tiempo cronometrado entre cada tubo de ensayo con el mismo número.
    \item Seguidamente, se halla la velocidad de reacción de cada mezcla y se iguala con la formula: $V=k\cdot[I^{-}]^{m}[S_2O_8^{2-}]^{n}$ para hallar los valores de m y n.
    \item Finalmente, la suma de m y n nos da el orden de la reacción.
    \end{enumerate}

    \subsection{Experimento 2: Efecto de la temperatura sobre la velocidad.}
    \begin{enumerate}
    \item Del experimento N°1, se escoje la mezcla con la menor velocidad de reacción.
    \item Depués, se repite el experimento anterior, pero antes de juntar las soluciones de ambos tubos de ensayo, se deja calentar en bañomaría por 5 minutos.
    \item Luego, se toma el tiempo en el que la mezcla se torna de color azul con un cronómetro y luego se halla la velocidad de reacción.
    \item Finalmente, se comparar con la velocidad obtenida en el primer experimento y observar si hay variaciones.
    \end{enumerate}

    \subsection{Experimento 3: Efecto de los catalizadores sobre la velocidad de reacción.}
    \begin{enumerate}
    \item Del experimento N°1, se escoje la mezcla que haya tenido el menor valor de velocidad de reacción y se prepara en 3 tubos de ensayo  $K_2$$S_2$$O_8$ con la misma concentración (los tubos se etiquetarán con las letras A, B y C para distinguirlos).  
    \item En el tubo A, se le agrega una gota de nitrato de cobre 1 M.
    \item En el tubo B, se le agrega una gota de nitrato de plata 1 M.
    \item En el tubo C, se le agrega una gota de agua.
    \item Luego, se mezcla $KI$ más almidón más tiosulfato más agua como en el experimento N°1.
    \item Después, se toma el tiempo de reacción de cada una de las mezclas con sus respectivos catalizadores y ver si producieron algún efecto en la velocidad de la reacción.
    \end{enumerate}

\end{document}