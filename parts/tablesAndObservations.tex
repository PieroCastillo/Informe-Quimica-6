\documentclass[../main.tex]{subfiles}

\begin{document}
\subsection{Observaciones}
\begin{itemize}
    \item En el experimento #3 la solución se tornó de color amarillo ocre en 4.10 segundos,
    luego cambió a azul claro en 5.5 segundos y finalmente se tornó totalmente 
    en azul oscuro en 6 segundos. 
\end{itemize}

\subsection{Tablas de Datos}
\begin{table}[H]
    \centering
    \begin{tabular}{c|c}
        \hline
        Ensayo & $\Delta t$ (s)\\
        \hline
        #1     & 16.60\\
        #2     & 16.01\\
        #3     & 60.23\\
        \hline
    \end{tabular}
    \caption{Datos experimentales del Experimento 1}
\end{table}
\begin{table}[H]
    \centering
    \begin{tabular}{c|c}
        \hline
        Ensayo & $\Delta t$ (s)\\
        \hline
        #3     & 5.5\\
        \hline
    \end{tabular}
    \caption{Datos experimentales del Experimento 2}
\end{table}
\begin{table}[H]
    \centering
    \begin{tabular}{c|c|c|c}
        \hline
        Ensayo        & #1 ($H_2O$) & #2 ($AgNO_3$) & #3 ($Cu(NO_3)_2$) \\
        \hline
        $\Delta t$ (s)& 70.18       & 84.81         & 5.10\\
        \hline
    \end{tabular}
    \caption{Datos experimentales del Experimento 3}
\end{table}


\end{document}