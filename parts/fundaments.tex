\documentclass[../main.tex]{subfiles}

\begin{document}

\subsection{Cinética Química}
Es un área que se encarga del estudio de la rapidez con que ocurre una reacción 
química. Cinética hace referencia al cambio en la concentración de un reactivo o 
de un producto con respecto del tiempo.
\subsection{Ley de la Rapidez} 
La rapidez de una reacción es proporcional a la 
concentración de reactivos y la constante de proporcionalidad "k" recibe el 
nombre de constante de rapidez. Esta ley expresa la relación de la rapidez de 
una reacción gracias a la constante de rapidez y la concentración de los 
reactivos, elevados a alguna potencia. 
Tenemos así:
\begin{equation} \nonumber
    a\,A + b\,B \rightarrow c\,C + d\,D
\end{equation}
Entonces la ley de rapidez tendrá la forma:
\begin{equation} \label{velocity_eq}
    V = k\cdot[A]^x[B]^y
\end{equation}

Los exponentes $ x $ y $ y $ especifican las relaciones entre las concentraciones 
de los reactivos A y B además de la rapidez de la reacción. Al sumar estos 
exponentes, obtenemos el orden de reacción global, que se define como la suma 
de los exponentes a los que se elevan todas las concentraciones de reactivos que 
aparecen en la ley de rapidez que escribimos anteriormente. El orden de una 
reacción siempre se define en términos de las concentraciones de los reactivos 
(no de los productos).

Las leyes de la rapidez siempre se determinan en forma experimental. A partir 
de las concentraciones de los reactivos como notamos en la ecuación \ref{velocity_eq}, 
además de la rapidez inicial es posible determinar el orden de una reacción y, 
por tanto, la constante de rapidez de la reacción.
\subsection{Técnicas con Yodo}
\subsubsection{Yodometría} Es un método indirecto en que los oxidantes se determinan 
haciéndolos reaccionar con un exceso de yoduro; el yodo liberado se valora en 
disolución débilmente ácida con un reductor patrón.\\
Los reactivos generalmente son tiosulfato del sodio como el titulante, el almidón 
como indicador que forma el complejo azulado con las moléculas del yodo, y 
un compuesto del yodo.
\subsubsection{Yodimetría} A diferencia de la yodometría, la yodimetría es un método 
directo en el que se utiliza una disolución modelo o patrón de yodo para valorar 
reductores fuertes, normalmente en disolución neutra o débilmente acidas.

\end{document}